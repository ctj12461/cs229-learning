\begin{answer}
	\\
	The answer to the first question is no.
	
	Assume that we select a specific range $(a, b)$ such that for all $i$'s in $\{i\mid i\in I_{a,b}\}$, $P(y^{(i)} = 1\mid x^{(i)}; \theta)$ are the same. Let's call that value $p$, so
	
	$$
	\frac{\sum_{i\in I_{a,b}} P\left(y^{(i)}=1|x^{(i)};\theta\right)} {{|\{i\in I_{a,b}\}|}}
	= \frac{\sum_{i\in I_{a,b}} p} {{|\{i\in I_{a,b}\}|}}
	= p
	$$
	
	Since $p\in (a,b)$ can't be $0$ or $1$ and the model is perfectly calibrated, the numerator of right hand side must be the sum of $0$'s and $1$'s, making right hand side a value in $(0, 1)$. So $y^{(i)}$ can be a either positive or negative example. But $p$ can only lies in either $(0, 0.5)$ or $(0.5, 1)$. Hence, there is only one possible outcome, $0$ or $1$, which implies inaccuracy when predicting these $x^{(i)}$ with two kinds of labels.
	
	In the converse, the answer is no, too.
	
	Choose $(a, b)=(0.5, 1)$, so
	
	$$
	\frac{\sum_{i\in I_{a,b}} \mathbb{I}\{y^{(i)} = 1\}}{|\{i\in I_{a,b}\}|} = \frac{\sum_{i\in I_{a,b}} 1}{|\{i\in I_{a,b}\}|} = 1
	$$
	
	Given $P\left(y^{(i)}=1|x^{(i)};\theta\right) \in (0, 1)$
	
	$$
	\frac{\sum_{i\in I_{a,b}} P\left(y^{(i)}=1|x^{(i)};\theta\right)} {{|\{i\in I_{a,b}\}|}}
	<\frac{\sum_{i\in I_{a,b}} 1} {{|\{i\in I_{a,b}\}|}} = 1
	$$
	
	Therefore
	
	$$
	\frac{\sum_{i\in I_{a,b}} P\left(y^{(i)}=1|x^{(i)};\theta\right)} {{|\{i\in I_{a,b}\}|}} \ne \frac{\sum_{i\in I_{a,b}} \mathbb{I}\{y^{(i)} = 1\}}{|\{i\in I_{a,b}\}|}
	$$
\end{answer}
