\clearpage
\item \subquestionpoints{7} For this part of the problem only, you may
  assume $n$ (the dimension of $x$) is 1, so that $\Sigma = [\sigma^2]$ is
  just a real number, and likewise the determinant of $\Sigma$ is given by
  $|\Sigma| = \sigma^2$.  Given the dataset, we claim that the maximum
  likelihood estimates of the parameters are given by
  \begin{eqnarray*}
    \phi &=& \frac{1}{m} \sum_{i=1}^m 1\{y^{(i)} = 1\} \\
\mu_{0} &=& \frac{\sum_{i=1}^m 1\{y^{(i)} = {0}\} x^{(i)}}{\sum_{i=1}^m
1\{y^{(i)} = {0}\}} \\
\mu_1 &=& \frac{\sum_{i=1}^m 1\{y^{(i)} = 1\} x^{(i)}}{\sum_{i=1}^m 1\{y^{(i)}
= 1\}} \\
\Sigma &=& \frac{1}{m} \sum_{i=1}^m (x^{(i)} - \mu_{y^{(i)}}) (x^{(i)} -
\mu_{y^{(i)}})^T
  \end{eqnarray*}
  The log-likelihood of the data is
  \begin{eqnarray*}
\ell(\phi, \mu_{0}, \mu_1, \Sigma) &=& \log \prod_{i=1}^m p(x^{(i)} , y^{(i)};
\phi, \mu_{0}, \mu_1, \Sigma) \\
&=& \log \prod_{i=1}^m p(x^{(i)} | y^{(i)}; \mu_{0}, \mu_1, \Sigma) p(y^{(i)};
\phi).
  \end{eqnarray*}
By maximizing $\ell$ with respect to the four parameters,
prove that the maximum likelihood estimates of $\phi$, $\mu_{0}, \mu_1$, and
$\Sigma$ are indeed as given in the formulas above.  (You may assume that there
is at least one positive and one negative example, so that the denominators in
the definitions of $\mu_{0}$ and $\mu_1$ above are non-zero.)

\ifnum\solutions=1 {
  \begin{answer}
	\\
	Prove $\phi = \frac{1}{m} \sum_{i=1}^m 1\{y^{(i)} = 1\}$:
	
	\begin{align*}
		\frac{\partial}{\partial \phi} \ell(\phi, \mu_0, \mu_1, \Sigma) & = \sum_{i = 1}^m \frac{\partial}{\partial \phi} (\ln p(x^{(i)} | y^{(i)}; \mu_{0}, \mu_1, \Sigma) + \ln p(y^{(i)}))\\
		& = \sum_{i = 1}^m \frac{\partial}{\partial \phi} \ln p(y^{(i)})\\
		& = \sum_{i = 1}^m \frac{y^{(i)} \phi^{y^{(i)} - 1} (1 - \phi)^{1 - y^{(i)}} - (1 - y^{(i)}) (1 - \phi)^{-y^{(i)}} \phi^{y^{(i)}}}{\phi^{y^{(i)}} (1 - \phi)^{1 - y^{(i)}}}\\
		& = \sum_{i = 1}^m \left(\frac{y^{(i)}}{\phi} - \frac{1 - y^{(i)}}{1 - \phi}\right)\\
		& = \sum_{i = 1}^m \frac{y^{(i)} - \phi}{\phi (1 - \phi)}\\
		& = \frac{\sum_{i = 1}^m 1\{y^{(i)} = 1\} - m\phi}{\phi (1 - \phi)}\\
		& = 0
	\end{align*}
	
	Therefore $\phi = \frac{1}{m} \sum_{i = 1}^{m} 1\{y^{(i)} = 1\}$.
	
	Prove $\mu_{0} = \frac{\sum_{i=1}^m 1\{y^{(i)} = {0}\} x^{(i)}}{\sum_{i=1}^m 1\{y^{(i)} = {0}\}} \\$
	
	\begin{align*}
		\frac{\partial}{\partial \mu_0} \ell(\phi, \mu_0, \mu_1, \Sigma) & = \sum_{i = 1}^m \frac{\partial}{\partial \mu_0} \ln p(x^{(i)} | y^{(i)}; \mu_{0}, \mu_1, \Sigma)\\
		& = \sum_{i,y^{(i)} = 0} \frac{\partial}{\partial \mu_0} p(x^{(i)} | y^{(i)} = 0; \mu_{0}, \mu_1, \Sigma)\\
		& = \sum_{i,y^{(i)} = 0} \frac{\partial}{\partial \mu_0} \ln \frac{1}{\sqrt{2\pi\Sigma}} \exp\left(-\frac{1}{2} \Sigma^{-1} (x^{(i)} - \mu_0)^2\right)\\
		& = \sum_{i,y^{(i)} = 0} \frac{\partial}{\partial \mu_0} [-\frac{1}{2}\Sigma (x^{(i)} - \mu_0)^2 ] \\
		& = \sum_{i,y^{(i)} = 0} \Sigma (x^{(i)} - \mu_0)\\
		& = \Sigma \sum_{i = 1}^m 1\{y^{(i)} = 0\}(x^{(i)} - \mu_0)\\
		& = 
	\end{align*}
	
	Therefore $\mu_{0} = \frac{\sum_{i=1}^m 1\{y^{(i)} = {0}\} x^{(i)}}{\sum_{i=1}^m 1\{y^{(i)} = {0}\}}$.
	
	For $\mu_1$, it can be proved in the same approach.
	
	Prove $\Sigma = \frac{1}{m} \sum_{i=1}^m (x^{(i)} - \mu_{y^{(i)}}) (x^{(i)} - \mu_{y^{(i)}})^T$:
	
	$$
	\begin{aligned}
		\frac{\partial}{\partial\Sigma} \ell(\phi, \mu_0, \mu_1, \Sigma) & = \sum_{i = 1}^m \frac{\partial}{\partial\Sigma} \ln\frac{1}{\sqrt{2\pi\Sigma}} \exp\left(-\frac{(x^{(i)} - \mu_{y^{(i)}})^2}{2\Sigma}\right)\\
		& = \sum_{i = 1}^m \frac{\partial}{\partial\Sigma} \left[\ln\frac{1}{\sqrt{2\pi}} - \frac{1}{2}\ln\Sigma - \frac{(x^{(i)} - \mu_{y^{(i)}})^2}{2\Sigma} \right]\\
		& = \sum_{i = 1}^m \frac{-\Sigma + (x^{(i)} - \mu_{y^{(i)}})^2}{2\Sigma^2}\\
		& = 0
	\end{aligned}
	$$
	
	Therefore $\Sigma = \frac{1}{m} \sum_{i=1}^m (x^{(i)} - \mu_{y^{(i)}})^2 = \frac{1}{m} \sum_{i=1}^m (x^{(i)} - \mu_{y^{(i)}}) (x^{(i)} - \mu_{y^{(i)}})^T$
	
\end{answer}
} \fi
